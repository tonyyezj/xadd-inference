%%HYBRID SYSTEMS

% We focus on controllability in stochastic systems... few results
% here except in chance-constrained systems where only
% approximate solutions are determined or dynamics
% is restricted to be Gaussian with state-independent noise.
% We argue that finding optimal solutions with exact guarantees
% on controllability in systems with state-dependent noise is
% crucial in settings like autonomous vehicles, satellite manuevers,
% and environmental control systems.

This work extends results in HMDP in AI
~\cite{boyan01,feng04,li05,kveton06,phase07,hao09,sdp_aaai} and hybrid
system control literature ~\cite{Henzinger:1997,Hu:2000,DeSHee:2009}
to handled state-dependent noise.

In the hybrid control literature, a challenging topic is to solve the
controllability problem that is NP hard \cite{Blondel:1999}. A hybrid
system is called hybrid controllable if, for any pair of valid states,
there exists at least one permitted control sequence (correct
control-laws) between them \cite{Tittus:1998,Yang:2007}.  Another
challenging topic for stochastic hybrid systems, a class of hybrid
systems that allows uncertainty, is tried to maximize the probability
that the execution will remain in safe states as long as
possible \cite{Hu:2000}.  This work is related with both topics,
however we want to answer a slightly different question, called the
robust controllability problem: what states have a policy to achieve a
goal (that can be modeled as a reward or cost function) with high
certainty over some horizon?  To the authors’ knowledge, in the
control area there are few results to answer a similar question except
in the chance-constrained predictive stochastic sub-area, that finds
the optimal sequence of control inputs subject to the constraint that
the probability of failure must be below a user-specified
threshold \cite{Blackmore:2011}. However all the previous work in this
sub-are is focused on linear systems subject to Gaussian uncertainty
and state-independence
noise \cite{Schwarm:1999,Li:2002,Ono:2008,Blackmore:2011} or resort to
approximation techniques \cite{Blackmore:2010}.  We note that 
our approach is not approximate and can optimally solve problems with
state-dependent noise in a receding horizon control framework that
answers the robust controllability question for the first time for such
problems.

 
 




